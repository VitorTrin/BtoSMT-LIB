B é um método formal usado para a modelagem de especificações e desenvolvimento rigoroso de software. BETA (Bbased testing approach) é uma ferramenta de geração de testes a partir de uma especificação em B. Uma parte do processo de criação de testes em BETA é, dado um teste, obter um conjunto de valores válidos para as entradas do teste, ou a informação que o teste não possui valores válidos. A ferramenta utilizada atualmente para encontrar valores válidos apresenta um problema de explosão de estados na geração de modelos. Como alternativa, buscamos o auxílio de resolvedores SMT (Satisfiability modulo theories), ferramentas capazes de resolver problemas de satisfabilidade em diversas combinações de teorias para obter esse conjunto de valores válidos.