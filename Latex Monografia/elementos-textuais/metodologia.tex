\chapter{Metodologia}\label{chp:LABEL_CHP_3}

Foram escolhidas ferramenta com um grande número de usuários e ferramentas usadas em casos parecidos em outras linguagens. Essas ferramentas são {log}, Alt-Ergo, CVC4 e Z3. Foram levantados então dados preeliminares dessas ferramentas e também de ProB, a ferramenta usada atualmente por BETA, baseados apenas na documentação de cada um deles,  nas categorias de:
\begin{itemize}
	\item Teorias suportadas: Se a ferramenta é capaz de lidar com todas as teorias envolvidas no teste;
	\item Documentação: Em termos de relevância para um usuário, amplitude e disponibilidade e, pois ela precisa ser informativa para quem for usar a ferramenta e não só para desenvolvedores, precisa descrever todo o uso, da instalação a execução, e precisa ser de livre e fácil acesso.Quanto melhor a documentação, mais fácil a interação com a ferramenta. Uma ferramenta com documentação fraca ou incompleta obriga a tentativa e erro.	
	\item Em desenvolvimento: Aplicações em desenvolvimento são capazes de melhorar e possuem pouca probabilidade de serem abandonados. É possível para sugerir novas melhorias para uma ferramenta em desenvolvimento para melhor atender nossas necessidades.
	\item Retorna casos que satisfazem:  Se é capaz de, dado as restrições do teste, retornar qual o conjunto de valores que satisfazem essas restrições e permitem a avaliação do teste. Esse é  um ponto principal para o nosso uso. Caso não haja o retorno dos casos que satisfazem, a ferramenta não possui utilidade.
	\item Usado por ferramentas de B: Indica que a compatibilidade com B já foi alcançada.
	\item Tamanho da base de usuários: Ferramentas com base de usuários maiores possuem menos chances de serem abandonados e tendem a se desenvolver mais rápido.
	\item Se possui API e sua linguagem: Caso possua uma API, é possível que Beta possa se comunicar com ele de modo transparente para o usuário.
	\item Linguagem de entrada: Modo de entrada padrão.
\end{itemize}