\chapter{Resultados dos Dados Preeliminares}\label{chp:LABEL_CHP_4}

\begin{table}[!h]
  \centering
  \begin{tabular}{ | l | p{10cm} |}
    \hline
    	  Nome & \{log\} \\ \hline
      Descrição & Uma língua de programação com restrições lógicas baseada em Prolog voltada para manipulação de conjuntos. \\ \hline
	  Website & \url{http://people.dmi.unipr.it/gianfranco.rossi/setlog.Home.html} \\ \hline
	  Teorias Suportadas & Sets, Integers. \\ \hline
	  Documentação & Possui exemplos, manual em \url{http://people.dmi.unipr.it/gianfranco.rossi/SETLOG/manual_4_9_1.pdf} \\ \hline
	  Em desemvolvimento & Último update 01/17 no beta, último estável 04/2016. \\ \hline
	  Retorna casos que satisfazem & Sem informação. \\ \hline
	  Usado por ferramentas de B & Nenhuma conhecida. \\ \hline
	  Tamanho da base de usuários & Nenhuma conhecida. \\ \hline
	  Possui API e sua linguagem & Nenhuma conhecida.  \\ \hline
	  Linguagem de entrada & Baseada em Prolog. \\ \hline	
  \end{tabular}
  \caption{Dados preeliminares de \{log\} }
  \label{tab:LABEL_TAB_1}
\end{table}

\begin{table}[!h]
  \centering
  \begin{tabular}{ | l | p{10cm} |}
    \hline
	Nome & Alt-Ergo \\ \hline
    Site & \url{http://alt-ergo.lri.fr/} \\ \hline 
    Teorias Suportadas & Free theory of quality with uninterpreted symbols, linear arithmetic over integers and rationals, non-linear arithmetic, polymorphic functional arrays, enumerated datatypes, record datatypes, associative and commutative (AC) symbols, fixed-size bit-vectors.\\ \hline
    Licença & CeCILL-C. \\ \hline
    Documentação & \url{https://github.com/OcamlPro/alt-ergo/blob/master/INSTALL.md}, muito voltada para instalação e funcionamento interno \\ \hline
    Em desenvolvimento & Última release estável 21/11/2016, sem commits desde então. \\ \hline
    Retorna casos que satisfazem & Sem informação. \\ \hline
    Usado por ferramentas de B & Atelier-B, Rodin \\ \hline
    Tamanho da base de usuários & Mais de 20 estrelas no github, usado na plataforma Why3 \\ \hline
    Possui API e sua linguagem & Sem informação. \\ \hline
    Linguagem de entrada & Linguagem Própria. \\ \hline
  \end{tabular}
  \caption{Dados preeliminares de Alt-Ergo }
  \label{tab:LABEL_TAB_2}
\end{table}

\begin{table}[!h]
  \centering
  \begin{tabular}{ | l | p{10cm} |}
    \hline
	Nome & CVC4 \\ \hline
    Site & \url{http://cvc4.cs.stanford.edu/} \\ \hline 
    Teorias Suportadas & Equality over free (aka uninterpreted) function and predicate symbols, real and integer linear arithmetic, bit-vectors, arrays, tuples, records, user-defined inductive datatypes, strings, finite sets, separation logic.\\ \hline
    Licença & CVC4 é All rights reserved, o código fonte é BSD. \\ \hline
    Documentação & Boa para API, necessário o uso da documentação do CVC3 na linguagem própria, terceriza parte da documentação SMT-LIB \\ \hline
    Em desenvolvimento & Sim, múltiplos commits diários, releases estáveis com uma relativa regularidade \\ \hline
    Retorna casos que satisfazem & Sim. \\ \hline
    Usado por ferramentas de B & Rodin \\ \hline
    Tamanho da base de usuários & Em 28/08, mais de 101 perguntas com a tag CVC4 no stackoverflow e 168 estrelas no repositório do github. \\ \hline
    Possui API e sua linguagem & Possui uma API em C++. \\ \hline
    Linguagem de entrada & Possui uma linguagem própria e aceita SMT-LIB \\ \hline
  \end{tabular}
  \caption{Dados preeliminares de CVC4 }
  \label{tab:LABEL_TAB_3}
\end{table}